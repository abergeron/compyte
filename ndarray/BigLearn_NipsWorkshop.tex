\documentclass{article} % For LaTeX2e
\usepackage{nips11submit_e,times}
\usepackage{natbib}
%\documentstyle[nips10submit_09,times,art10]{article} % For LaTeX 2.09


\title{A Common GPU N dimensions array}


\author{
Frédéric Bastien\thanks{ Use footnote for providing further information
about author (webpage, alternative address)---\emph{not} for acknowledging
funding agencies.} \\
D\'epartement d'Informatique et de Recherche Op\'erationelle\\
Universit\'e de Montr\'eal\\
Montréal, Canada \\
\texttt{nouiz@nouiz.org} \\
\And
Coauthor \\
Affiliation \\
Address \\
\texttt{email} \\
\AND
Coauthor \\
Affiliation \\
Address \\
\texttt{email} \\
}

% The \author macro works with any number of authors. There are two commands
% used to separate the names and addresses of multiple authors: \And and \AND.
%
% Using \And between authors leaves it to \LaTeX{} to determine where to break
% the lines. Using \AND forces a linebreak at that point. So, if \LaTeX{}
% puts 3 of 4 authors names on the first line, and the last on the second
% line, try using \AND instead of \And before the third author name.

\newcommand{\fix}{\marginpar{FIX}}
\newcommand{\new}{\marginpar{NEW}}

\nipsfinalcopy % Uncomment for camera-ready version

\begin{document}


\maketitle

\begin{abstract}
Currently there is multiple array/matrix/n-dimensional array
object that exist on the gpu.  This hinder sharing gpu code and cause
duplicate development work. This paper present a proposition and a
first version for a common gpu n-dimensional array(tensor) on the GPU. It will
work on CUDA and OpenCL. It will be usable in python and C++.
\end{abstract}

\section{Introduction}
Motivation:
    * Currently there is at least 4 different gpu array in python
          o CudaNdarray(Theano), GPUArray(pycuda) and CUDAMatrix(cudamat), GPUArray(pyopencl), ...
          o There is even more if we include other language.
    * They are incompatible
          o None have the same properties and interface
    * All of them are a subset of numpy.ndarray on the gpu!

Lack of Standard Creates Problems

    * Duplicate work
          o GPU code is harder/slower to do correctly and fast than on the CPU/python
    * Harder to port/reuse code
    * Harder to find/distribute code
    * Divides development work


\subsection{Pitfall to Avoid}
    * Start alone
          o We need different people/groups to ``adopt'' the new GpuNdArray
    * Too simple - other projects won't adopt as it won't fit there need.
    * Too general - other projects will implement ``light'' version... and not adopt
          o Having an easy way to convert/check conditions as numpy could alleviate this.


\subsection{Goal}
    * Make a n dimensionals array on the gpu that support strides
    * Make the python interface VERY similar to numpy.ndarray
          o Easier to attract other peole from python community
    * Have the base object in C to allow collaboration with more project.
          o We want people from C, C++, ruby, R, ... all use the same base Gpu ndarray.
    * Be compatible with CUDA and OpenCL


The option choosed is to have a n-dimensional array(tensor) with easy
check/conversion to easily allow supporting only a subset!

Make a common GPU ndarray(matrix/tensor or n dimensions) that can be reused by all project
WARNING

This is in early development. So all what you read is discutable. If you have idea send them to the mailing list!
Mailing list

    * Development/user mailing list: http://lists.tiker.net/listinfo/gpundarray
    * Announce mailing list(low volume): http://lists.tiker.net/listinfo/gpundarray-announce

Multiple backend:

Current behavior not wanted

    * No CPU code generated from the python interface(for PyOpenCL and PyCUDA) Gpu code are ok.


\section{Existing Implementation}
\subsection{Theano(CudaNdarray)}
Theano~\citep{bergstra+al:2010-scipy}
tensor with strides, float32 only, cuda only
\subsection{PyCUDA/PyOpenCL(GPUArray)}
contiguous memory only, cuda or opencl only(similar interface)?
can use both in the same programme. all dtype
\subsection{cudamat(CUDAMat)}
cuda only, float32 only I think
\subsection{Gnumpy}
on top of cudamat.
\subsection{Trust}
vector only. CUDA only, all dtype

\section{Current implementation}


\subsection{Functinality}
fct implemented language dependent
cuda
malloc/free
memcopy
memset
from cpu to gpu
from gpu to cpu

gpu agnos:
sub tensor
deepcopy
view
zeros
empty

probably gpu agnos, not sure:
elemwise
elemwise dimensions collapsing


todo:
assignation
reshape
n dimensional tranpose
dot22,gemm
reduce along all dimensions, along one axis or a list of dimensions
reduce dimensions collapsing

\subsection{Files structure}
File structure that allow using Cuda and PyOpenCL use the same code for some functionality.

\section{Benchmark}
\section{Futur Plan}
\section{Conclusion}

\section{Citations, figures, tables, references}
\label{others}

These instructions apply to everyone, regardless of the formatter being used.

\subsection{Citations within the text}

Citations within the text should be numbered consecutively. The corresponding
number is to appear enclosed in square brackets, such as [1] or [2]-[5]. The
corresponding references are to be listed in the same order at the end of the
paper, in the \textbf{References} section. (Note: the standard
\textsc{Bib\TeX} style \texttt{unsrt} produces this.) As to the format of the
references themselves, any style is acceptable as long as it is used
consistently.

\begin{table}[t]
\caption{Sample table title}
\label{sample-table}
\begin{center}
\begin{tabular}{ll}
\multicolumn{1}{c}{\bf PART}  &\multicolumn{1}{c}{\bf DESCRIPTION}
\\ \hline \\
Dendrite         &Input terminal \\
Axon             &Output terminal \\
Soma             &Cell body (contains cell nucleus) \\
\end{tabular}
\end{center}
\end{table}

\section{Preparing PostScript or PDF files}

\subsection{Margins in LaTeX}
 
Most of the margin problems come from figures positioned by hand using
\verb+\special+ or other commands. We suggest using the command
\verb+\includegraphics+
from the graphicx package. Always specify the figure width as a multiple of
the line width as in the example below using .eps graphics
\begin{verbatim}
   \usepackage[dvips]{graphicx} ... 
   \includegraphics[width=0.8\linewidth]{myfile.eps} 
\end{verbatim}
or % Apr 2009 addition
\begin{verbatim}
   \usepackage[pdftex]{graphicx} ... 
   \includegraphics[width=0.8\linewidth]{myfile.pdf} 
\end{verbatim}
for .pdf graphics. 
See section 4.4 in the graphics bundle documentation (http://www.ctan.org/tex-archive/macros/latex/required/graphics/grfguide.ps) 
 
A number of width problems arise when LaTeX cannot properly hyphenate a
line. Please give LaTeX hyphenation hints using the \verb+\-+ command.


\subsubsection*{Acknowledgments}

Use unnumbered third level headings for the acknowledgments. All
acknowledgments go at the end of the paper. Do not include 
acknowledgments in the anonymized submission, only in the 
final paper. 

\bibliography{strings,strings-shorter,ml,aigaion-shorter}
\bibliographystyle{plain}

\end{document}
